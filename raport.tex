
\input{documentTemplate.tex}
\newcommand{\labdate}{28 listopada 2020}
\newcommand{\temat}{Mikroskop MET-3}

\begin{document}
    
\input{tabela.tex}


\section{Cel ćwiczenia}

Celem ćwiczenia było zapoznanie się z~budową mikroskopu metalograficznego MET-3 do badań w~świetle odbitym, opanowanie sposobu jego ustawiania i~obsługi oraz zastosowanie mikroskopu do badań jakościowych i~ilościowych zgładów.

\section{Wprowadzenie}



\section{Aparatura, osprzęt i~materiały}



\section{Wyniki i~ich dyskusja}

\subsection{Przygotowanie mikroskopu do badań.}

Jako przygotowanie mikroskopu do badań, w~pierwszym kroku ustawiliśmy oświetlenie wg. zasady Kohlera.
W tym celu postępowaliśmy według następujących kroków:
\begin{itemize}
    \item Przygotowaną próbkę umieściliśmy na stoliku.
    \item Pod badaną próbką ustawiliśmy obiektyw 8x oraz wsunęliśmy do nasadki tubusa okular.
    \item Włączyliśmy oświetlenie.
    \item Otworzyliśmy przesłonę pola za pomocą pierścienia.
    \item Patrząc przez okular i~obracając pokrętką ruchu zgrubnego, później ruchu drobnego ustawiliśmy ostrość obserwowanego obrazu badanej próbki.
    \item Zmniejszyliśmy otwór przesłony pola, w~taki sposób, aby jej obraz był nieco mniejszy od pola widzenia.
    \item Wyjęliśmy okular z~tubusa.
    \item Nieznacznie przesuwając oświetlacz względem podstawy uzyskaliśmy ostry obraz włókna żarówki.
    \item Zmniejszyliśmy do około 2/3 wielkości średnicy wyjściowej obiektywu obraz przysłony aperturowej.
    \item Ponownie wsunęliśmy okular do tubusa i~ponownie ustawiliśmy ostrość obrazu, tym razem tylko za pomocą pokrętki ruchu drobnego.
\end{itemize}

\subsection{Opis ćwiczenia.}

Następnym krokiem po przygotowaniu mikroskopu do badań było zidentyfikowanie 6 różnych próbek (zdjęcie identyfikacji w~załącznikach), stworzonych z~innych materiałów (grafit, EGH, SiC, SiC-Si, ferryt, płytka krzemowa) i~wykonanie schematycznego rysunku czterech z~nich pod powiększeniem 8x oraz czterech pod powiększeniem 40x. Kolejnym naszym zadaniem było zbadanie udziału objętościowego faz w~ferrycie. Aby było to możliwe, wykorzystaliśmy specjalne obiektywy z~wykalibrowaną siatką. 
Zliczaliśmy liczbę przecięć linii poziomych i~pionowych w~każdej z~faz z~osobna, na powiększeniu 8x, a~następnie bez poruszania próbką, na powiększeniu 40x.

\subsection{Obliczenia.}

Dokonaliśmy obliczeń powiększenia mikroskopu, z~wykorzystaniem powiększeń 8x i~40x. W~tym celu wykorzystaliśmy wzór:

$$p=p_{gł}\cdot p_{ob}\cdot p_{ns}\cdot p_{ok}$$

gdzie:
\newline
$p_{gł}$ - powiększenie głowicy
\newline
$p_{ob}$ - powiększenie obiektywu
\newline
$p_{ns}$ - powiększenie nasadki okularu
\newline
$p_{ok}$ - powiększenie okularu
\newline

Dla powiększenia 8x: 
$$p_{8x}=1.25\cdot 8\cdot 1\cdot 10=100$$

Dla powiększenia 40x:
$$p_{40x}=1.25\cdot 40\cdot 1\cdot 10=500$$

Schematyczne rysunki, wraz z~załączonymi zdjęciami ich realnych odpowiedników, znajdują się w~załącznikach wyników (na końcu sprawozdania).
\newline

Kolejnym naszym zadaniem było policzenie udziału objętościowego porów w~ferrycie. Badanie to wykonywaliśmy dwukrotnie, przez dwie różne osoby, w~dwóch różnych miejscach próbki. Zliczone liczby porów w~ferrycie przedstawia poniższa tabela. Aby wyliczyć udział objętościowy, wykorzystaliśmy wzór:
$$U_{obj}=\cfrac{L_p}{L_l} \cdot 100\%$$
\newline

gdzie:
\newline
$L_p$ - Liczba porów na skrzyżowaniu linii
\newline
$L_l$ - Liczba skrzyżowanych linii

% Please add the following required packages to your document preamble:
% \usepackage{graphicx}
\begin{table}[H]
    \caption{Liczba porów zliczona przez osobę numer 1.}
    \resizebox{\textwidth}{!}{%
    \begin{tabular}{|c|c|c|c|}
    \hline
    \multicolumn{2}{|c|}{Powiększenie 8x}           & \multicolumn{2}{c|}{Powiększenie 40x}           \\ \hline
    Lp.                 & Ilość porów w danej linii & Lp.                 & Ilość porów w danej linii \\ \hline
    1                   & 4                         & 1                   & 5                         \\ \hline
    2                   & 4                         & 2                   & 3                         \\ \hline
    3                   & 5                         & 3                   & 4                         \\ \hline
    4                   & 6                         & 4                   & 3                         \\ \hline
    5                   & 5                         & 5                   & 2                         \\ \hline
    6                   & 8                         & 6                   & 2                         \\ \hline
    7                   & 5                         & 7                   & 1                         \\ \hline
    8                   & 5                         & 8                   & 3                         \\ \hline
    9                   & 4                         & 9                   & 4                         \\ \hline
    10                  & 3                         & 10                  & 2                         \\ \hline
    11                  & 6                         & 11                  & 2                         \\ \hline
    Suma                & 37                        & Suma                & 31                        \\ \hline
    Udział objętościowy & 47\%                      & Udział objętościowy & 26\%                      \\ \hline
    \end{tabular}%
    }
    \end{table}
% Please add the following required packages to your document preamble:
% \usepackage{graphicx}
\begin{table}[H]
    \caption{Liczba porów zliczona przez osobę numer 2.}
    \resizebox{\textwidth}{!}{%
    \begin{tabular}{|c|c|c|c|}
    \hline
    \multicolumn{2}{|c|}{Powiększenie 8x}           & \multicolumn{2}{c|}{Powiększenie 40x}           \\ \hline
    Lp.                 & Ilość porów w danej linii & Lp.                 & Ilość porów w danej linii \\ \hline
    1                   & 4                         & 1                   & 4                         \\ \hline
    2                   & 4                         & 2                   & 1                         \\ \hline
    3                   & 6                         & 3                   & 3                         \\ \hline
    4                   & 4                         & 4                   & 2                         \\ \hline
    5                   & 4                         & 5                   & 2                         \\ \hline
    6                   & 3                         & 6                   & 1                         \\ \hline
    7                   & 5                         & 7                   & 1                         \\ \hline
    8                   & 5                         & 8                   & 2                         \\ \hline
    9                   & 5                         & 9                   & 2                         \\ \hline
    10                  & 5                         & 10                  & 1                         \\ \hline
    11                  & 4                         & 11                  & 2                         \\ \hline
    Suma                & 49                        & Suma                & 21                        \\ \hline
    Udział objętościowy & 40\%                      & Udział objętościowy & 17\%                      \\ \hline
    \end{tabular}%
    }
    \end{table}

\section{Podsumowanie i~wnioski}

\section{Załączniki wyników}

\end{document}
